\section{Halo exchange}

\subsection{Global halo exchange}

This is required if self-similar (or periodic) boundary
conditions are preferred.
Self-similar BC are probably more useful than
fixed BC when grain statistics are important.
For example, if fixed BC are used grains touching
the boundary are not growing and evolving in the
same way as grains which border only other grains.
If sufficiently many grains are simulated, then
self-similar BC improve grain size statistics.
Otherwise the grains touching the model boundary
must be excluded from the analysis.

Global halo exchange is more complex than local.

\subsubsection{2D planes}

\subsubsection{1D edges}

The diagram in Fig. \ref{fig:cgca0}

\begin{figure}
\centerline{\includegraphics[width=\textwidth]{./cgca0.pdf}}
\caption{The same diagram is used for exchanging edges
along all three axes.
The coord. systems in the middle are to help understand
the array assignments.}
\label{fig:cgca0}
\end{figure}

\subsubsection{corners}

\begin{figure}
\centerline{\includegraphics[width=\textwidth]{./cgca1.pdf}}
\caption{The same diagram is used for exchanging corners
along all three axes.
The coord. systems in the middle are to help understand
the array assignments.}
\label{fig:cgca1}
\end{figure}
