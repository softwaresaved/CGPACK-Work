\section{\texttt{CG\_CPP}:
 calculation of the minimum angle (maximum projection)
 between a vector in a CA coord. system and a crystal
 vector}

\label{sec:cg_cpp}

\subsection{Purpose}

The aim of this routine is to find a crystallographic
direction (or a crystallographic plane having this
direction as a normal)
that is closest to the direction of the
applied vector in CA coord. system. If the applied
vector is the direction of the maximum principal
stress, then this plane is a candidate for cleavage
propagation.

\subsection{See also}

\hyperref[sec:cg_clvg]{\texttt{CG\_CLVG}},
\hyperref[sec:cg_csym]{\texttt{CG\_CSYM}}

\subsection{Interface}

\begin{verbatim}

! IN:
!        AVEC - A VECTOR IN CA COORD. SYSTEM
!           R - ROTATION TENSOR, DEFINING THE ORIENTATION OF A CRYSTAL
!        CVEC - CRYSTAL VECTOR, IN CRYSTAL COORD. SYSTEM
! OUT:
!     MAXPROJ - MAX PROJECTION OF THE AVEC ONTO CVEC AND ALL OTHER
!               CRYSTAL VECTORS DEFINING PLANES OF THE SAME FAMILY
!               AS CVEC.
!         CCA - THE CRYSTAL VECTOR IN CA COORD. SYSTEM, WHICH
!               PRODUCES THE MAX PROJ VALUE.
!
!**********************************************************************73
!
! ONELINE: MIN ANGLE BETWEEN A VECTOR AND A NORMAL TO A CRYSTAL PLANE
!
!**********************************************************************73

 SUBROUTINE CG_CPP(AVEC,R,CVEC,MAXPROJ,CCA)

 IMPLICIT NONE

 REAL(KIND=8),INTENT(IN) :: AVEC(3),R(3,3),CVEC(3)
 REAL(KIND=8),INTENT(OUT) :: MAXPROJ,CCA(3)

\end{verbatim}

\subsection{Details}

This routine is non-trivial only because of the
need to take crystal symmetry into account. Refer
to Fig. \ref{fig:cg_cpp} below.

\begin{figure}[htb]
\centering
\includegraphics[width=0.5\textwidth]{./cg_cpp.pdf}
\caption{Schematic diagram illustrating
\hyperref[sec:cg_cpp]{\texttt{CG\_CPP}}
routine. Refer to the main text for details.}
\label{fig:cg_cpp}
\end{figure}

Consider a CA array with its coord. system (superscript
CA). A vector $\vec{a}$ is defined in this coord. system.

Within the CA array there is a grain (crystal) defined by
its rotation tensor $\mathbf{R}$, which has the usual
meaning: a vector $\vec{c}$ in the crystal coord. system
is transformed into a vector in the CA coord. system as

\begin{equation}
 c^{CA}
  =
   \mathbf{R} c^{C}
 \label{eq:cpp:pub}
\end{equation}
%
where superscript $C$ means relative to the crystal
coord. system.

Both $\vec{a}$ and $\vec{c}$ are unit vectors.

The dot product of $\vec{a}$ and $\vec{c}$ is
obtained simply as (in tensor notation)

\begin{equation}
 \texttt{dot product }
  =
   a\cdot\mathbf{R}c
 \label{eq:cpp:pluck}
\end{equation}

However, due to crystal symmetry, $\vec{c}$ defines
a class of planes, not just one plane. To take this into
account we use rotational symmetry tensors, $\mathbf{R}^S$,
see \hyperref[sec:cg_csym]{\texttt{CG\_CSYM}}.
For cubic crystals there are 24 distinct $\mathbf{R}^S$
tensors, including the trivial $\mathbf{R=I}$.

So we get 24 dot products, of which we choose
the one with the maximum absolute value
for output

\begin{equation}
 \texttt{max dot product }
  =
   \max
    |
     a \cdot \mathbf{R} \mathbf{R}^S c
    |
 \label{eq:cpp:mumps}
\end{equation}

The rotational symmetry
tensor, producing the maximum dot product value,
$\mathbf{R}^S_{max}$ is used to calculate the
crystallographic plane, of type defined by $\vec{c}$,
that is closest to being normal to $\vec{a}$:
$\mathbf{R}^S_{max}c$.

